% !TeX root = ../main.tex
% Add the above to each chapter to make compiling the PDF easier in some editors.

\chapter{Conclusion}\label{chapter:conclusion}

This chapter concludes the work of this thesis, we first give a summary about the proposed software framework,  thesis flow and  main contribution of this work. Second, we demonstrate the results and  outcome of this thesis. Finally, we discuss the future work and enhancements for  this work. 

\section{Summary}
To sum up, we designed and presented a delay-tolerant and information-centric framework architecture for pervasive computing  that distributes, composes and executes flows which are representations of computations describing pervasive use cases. The framework uses  service discovery, allows sending and receiving of flows while accounting for their dependencies, required hardware resources, sensors and actuators even without an end-to-end path between senders and receivers. Moreover, it provides the execution environment for the flows and a user interface for publishing and designing flows. Further, the framework allows devices to communicate with each other and exchange data in a publish-subscribe manner which enables them to compose and have inter-relationships in addition to  being able to locally exchange data through databases hosted on the same device. \\

\noindent In this thesis, we first gave an introduction to the problem and our intent to design a pervasive framework for computation distribution, composition and execution even in challenged networks. Therefore in the background chapter, we researched the current accomplishments in the fields of pervasive computing, delay-tolerant and information-centric networking. After harnessing these concepts and architectures we came up with ideas and foundations which led us to design our framework architecture. However, in order to evaluate it, we needed concrete real life use cases for  requirements elicitation so that we can evaluate our framework against these requirements which were discussed in the system design chapter. We also explained the middleware implementation and used the extracted requirements to devise experimental use cases to evaluate our framework.\\

\noindent The main contribution of this work is both the framework design and the implementation of Maestro which acts as a middleman between the messaging system and the execution environment. It handles the flows necessities by attaching the dependencies, checks the flow meta-data in order to make sure the device meets desired requirements and has the required resources. Once the flow passes all checks, it deploys the flow to the execution environment. \\


\noindent The evaluation results of our  proof-of-concept implementation and system design showed the feasibility of our architecture. The use case experiments that we ran satisfied the  requirements which were devised from  real life use cases. We have proved the viability of distributing and executing flows with dependencies to smart devices with different resources and gadgets. Further, we presented that devices can exchange messages and compose different flows locally and globally. Beyond that, we confirmed that messages are delivered to challenged networks that does not have an end-to-end path. Finally, we provided the mean and standard deviation of the delays taken between sending and deploying messages to the execution environment 



\section{Future Work}
The framework can be extended and enhanced on several aspects:
\begin{itemize}
\item \textit{Streaming}: a possible extension to the framework is to allow live feed or footage to be streamed from one device to another, of course this could be done by composing flows in which one device captures a video and sends the frames to other devices. However, this might experience some delay, it would be more efficient to have a streaming API in the messaging system which can be exposed to both the middleware and execution environment.

\item \textit{Request-Response Mapping}: In the middleware implementation there is no way at the moment to map a published message sent as a request  with another sent as a response. More specifically, in the challenged networks experiment, we could not be sure which requests for image recognition sent from the Raspberry Pi lead to the successful  replies  sent from the NUC back to the Raspberry Pi. This is not the case with recognizing water bottles experiment because we sent one request each run and waited for the response. Unlike the challenged networks experiment, we sent a lot of requests and received less replies. This could possibly be done by forwarding the unique message identifier from the request to the reply thus mapping between them.

\item \textit{Security}: This thesis does  not focus on securing the communication between devices and ensuring that requests and deployments to the execution environment are authenticated. This could be enhanced by providing a layer of security in the messaging and deployment process.

\item \textit{Resources Discovery}: The current middleware implementation reads the resources from a specification file on the smart devices. It would be more dynamic and flexible if the middleware could discover the attached resources dynamically and sensitive to the addition or removal of gadgets. 


\end{itemize}