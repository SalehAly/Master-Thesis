% !TeX root = ../main.tex
% Add the above to each chapter to make compiling the PDF easier in some editors.

\chapter{Background \& Related Work}\label{chapter:background}

\section{Internet of Things}

Internet of things \textbf{IoT} is one of the most trending topics in the software industry. In general terms, IoT refers to a distributed network of connected devices equipped with context-aware gadgets that enables them to see, hear, think and act intelligently\cite{DAC:DAC2417}. They are also allowed to communicate and share knowledge, which make them smart, powerful and capable of acting independently. Thus, a device is qualified to be a smart one if: it has physical existence, can communicate, has unique global identifier, has a name and address, can sense the environment and has basic computing capabilities\cite{Miorandi20121497}. Currently the number of connected devices are estimated in billions, they aim to automate everything around us and are mainly targeted to increase life quality of home and business applications.


Exploiting the powerfulness of these smart devices a lot of concepts emerged:
\subsection{Distributed Wireless Sensor Networks}

\subsection{Pervasive Computing} 

\subsection{Fog Computing}



\section{Networking}
\subsection{Delay Tolerant Networking}
\subsection{Information Centric Networking}

\section{Used Platforms}
\subsection{Node-Red}
\subsection{SCAMPI}
\subsection{Raspberry Pi}
\subsection{Time-series Databases}

%\section{Illustrate what are the ideas and possible network mechanisms and protocols that could be used data transfer}
%\subsection{Server To Server }
%\subsection{Server To Device }
%\subsection{Device To Device }

