% !TeX root = ../main.tex
% Add the above to each chapter to make compiling the PDF easier in some editors.

\chapter{Introduction}\label{chapter:introduction}
%In this chapter, we introduce our intent and motivation for this work. Afterwards, we define the scope and goals  then give an overview of the thesis structure.
%\section{Motivation}
 The concept of Pervasive or Ubiquitous computing, in which devices are integrated with intelligent agents and expected to support human needs anytime and anywhere without their interference, existed long ago since 1991 in Mark Weiser's paper "The Computer for the 21st Century" \cite{weiser1991ubicomp}. Even though pervasive computing  was there for a long time, research and development in this area have recently flourished again following the  prosperity of wireless sensor networks, single board computers and embedded micro-processors in our daily objects. Today, pervasive devices are expected to act on their own, be context-aware and include intelligent agents to support human beings and increase their life quality while being indistinguishable from  everyday objects. In 1999 the term Internet of Things (IoT) appeared and since then it has been used to refer to networks of smart things and  ideas around establishing smart cities, homes and factories. Smart devices used in these contexts are heterogeneous and  expected to communicate and share knowledge with either each other or  the cloud.\\
 
 \noindent Despite the various applications for pervasive agents and IoT networks, there is still a lack of frameworks that could harmonize and orchestrate the deployment of different use cases to these networks. There are many reasons for this absence, first is the heterogeneity of devices, they differ in their computing capabilities. Second is the varying  availability of gadgets including sensors and actuators attached to the smart devices in the network. Third, the different nature of use cases and their dependencies develops an obstacle towards using a unified framework for most use cases. Fourth, providing a communication mechanism to the framework that works seamlessly and discovers other peer smart devices remains a challenge. Last but not least, the existence of challenged networks makes it rather hard to reach devices without an end-to-end path. Advances in research specifically  pervasive computing, IoT, Delay-Tolerant and Information-Centric networking in addition to current available platforms, have given us insights in order to pursue creating a framework for pervasive computing.\\
 
 \noindent The main goal of our work is to propose a framework architecture that can distribute  pervasive use case computations with their dependencies to all smart devices in a network with respect to their resources. Also, the framework allows peer discovery and communication for  smart devices without host names which makes it even more dynamic. The framework also allows computations to  be self-contained and have inter-relationships between each other in addition to providing the proper execution environment. Further, we extended the framework architecture to deliver the computations to smart devices even in challenged networks. \\
 
 \noindent Our framework architecture suggests a stack that is installed on each smart device
and composed of a delay-tolerant, information-centric and publish subscribe messaging
system. It handles the  communication between all smart devices and implements
service discovery at the stack bottom. Above the messaging system,  a middleware exists
that harmonizes and orchestrates computation dependencies, resources and
deployment. This is done by ensuring that each computation has its dependencies
attached, and verifies that the smart device has the required resources from hardware
requirement, sensors and actuators. Then the middleware decides whether to deploy
the computation or not. Finally on top of the stack, an execution environment that acts
as a host for all computations deployed from the middleware and as a user interface to
design and compose different use case computations.
 
\section{Scope and Goals}
The scope of this thesis is to present the design and architecture of a pervasive  framework for distributing, composing and executing computations which implement various use cases. The design is based on the concepts of delay-tolerant, information-centric networking, pervasive computing and executing computations on the network edge. Requirements for the framework design are extracted from real life use cases. We base our design on these requirements and also present a proof-of-concept implementation to show that the design is feasible and can be realized. Then, the implementation and design are evaluated by running several experiments each targeting a specific set of requirements.\\

	\noindent The framework will enable smart devices to communicate together through a publish-subscribe scheme even without an end-to-end path, thus sending and receiving data and computations.  It will also provide the user  an interface to design flows which are required to implement use cases as well as to publish flows either to all the smart devices in a network, a set or a specific device. In addition, it will allow the user to adjust the resources and attach dependencies to flows in order to compensate for the heterogeneous devices in the network and make sure flows will be executed successfully.\\

\noindent More specifically, the goal of this thesis is to provide a framework that enables the user to: i) develop a flow that portrays a pervasive use case, ii) choose the flow hardware requirements such as the computation power and the random access memory that fits the use case, iii) choose the required sensors and actuators needed by the flow, iv) attach necessary flow dependencies   in order to ensure the computation will run successfully, v) send the flow to all devices in a network, vi) ensures the flow is received by all targeted devices, vii) handle the deployment and execution of the flow if it satisfies the needed requirements and resources, viii) allow data communication between the framework devices using information-centric architecture, ix) compose flows by matching their the input and output, x) reach isolated devices in separate networks using delay-tolerant architecture. \\



 

\section{Thesis Structure}

This thesis is constructed in the following way; in the first chapter we present the problem along with a brief introduction to the proposed solution. The second chapter consists of the background information  that represent the underlying  concepts of this thesis in addition to the platforms used to implement the proposed framework. In the third chapter we define the theoretical terms, key concepts and possible solutions to the framework's main challenges. Afterwards in Chapter \ref{chapter:Approach}, we explain some of the real life use cases, requirements elicitation and the approach we took to design our framework architecture. Chapter \ref{chapter:implementation} describes the implementation details of the main contribution of this work besides the use cases implementation used in the evaluation chapter. Then, we evaluate the framework implementation and  architecture in Chapter \ref{chapter:Evaluation} by running several experimental use cases each satisfying a different set of requirements. Finally we conclude our work, summarize the results and also explain the room for improvement and future work in Chapter \ref{chapter:conclusion}.